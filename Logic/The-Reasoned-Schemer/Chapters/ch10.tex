\documentclass[letterpaper]{article}
\usepackage{slatex}
\begin{document}

\title{Chapter 10}

\author{(Typeset by Alexander L. Hayes)}

\maketitle

Now it is time to understand the core of $\equiv$, $\mathbf{fresh}$, $\mathbf{cond}^{e}$,
$\mathbf{run}$, $\mathbf{run^{*}}$, and $\mathbf{defrel}$.

Of course, we show the core of $\mathbf{conda}^{a}$ and $\mathbf{conda}^{u}$ as well.

Let's begin with $\equiv$. The definition of $\equiv$ relies on \textit{unify}, which
we shall discuss soon. But we'll need a few new ideas first.

\hrulefill

Here is how we create a unique\footnote{
  \textit{vector} creates a vector, a datatype distinct
  from pairs, strings, characters, numbers, Booleans, symbols, and \scheme'()'. Each use of
  \textit{var} creates a new one-element vector representing a unique variable. We ignore the
  vector's contents, instead distinguishing vectors by their address in memory. We could instead
  distinguish variables by their value, provided we ensure their values are unique (for example,
  using a unique natural number in each variable).
}
variable.

\begin{schemedisplay}
  (define (var name) (vector name))
\end{schemedisplay}

And here is a simple definition of \textit{var?}.

\begin{schemedisplay}
  (define (var? x) (vector? x))
\end{schemedisplay}

\hrulefill

We create three variables $u$, $v$, and $w$.

\begin{schemedisplay}
  (define u (var 'u))
  (define v (var 'v))
  (define w (var 'w))
\end{schemedisplay}

And here are the variables $x$, $y$, and $z$.

\begin{schemedisplay}
  (define x (var 'x))
  (define y (var 'y))
  (define z (var 'z))
\end{schemedisplay}

The pair \scheme{`(,z . a)} is an \textbf{association} of \scheme'a' with the variable
$z$.

A pair is an \textit{association} when the \textit{car} of that pair is a variable.
The \textit{cdr} of an association may itself be a variable or a value that contains
zero or more variables.

\begin{schemedisplay}
  (cdr `(,z . b))
  ; b.

  (cdr `(,z . (,x e ,y)))
  ; `(,x e ,y)
\end{schemedisplay}

The list \scheme{`((,z . oat) (,x . nut))} is a \textbf{substitution}.

What is a \textit{substitution}?

A \textit{substitution}\footnote{
  These substitutions are known as \textit{triangular} substitutions. For more on these
  substitutions see Franz Baader and Wayne Snyder. ``Unification theory,'' Chapter 8
  of \textit{Handbook of Automated Reasoning}, edited by John Alan Robinson and Andrei
  Voronkov. Elsevier Science and MIT Press, 2001.
} is a special kind of list of associations. In the substitution \scheme{`((,x . ,z))},
what does the substitution \scheme{`(,x . ,z)} represent?

In a substitution, an association whose \textit{cdr} is also a variable represents the
fusing of that association's two variables.

Here is \textit{empty-s}: the substitution that contains no associations:

\begin{schemedisplay}
  (define empty-s '())
\end{schemedisplay}

What is \textit{empty-s}? It's the substitution that contains no associations.

Is \scheme{`((,z . a) (,x . ,w) (,z . b))} a substitution? Not here, since our
substitution cannot contain two or more associations with the same \textit{car}.

What is the value of the following?

\begin{schemedisplay}
  (walk z
    `((,z . a) (,x . ,w) (,y . ,z)))
\end{schemedisplay}

\scheme'a', because we look up $z$ in the substitution (\textit{walk}'s second
argument) to find its association, \scheme{`(,z . a)}, and \textit{walk} produces
this association's \textit{cdr}, \scheme'a', since $a$ is not a variable.

\end{document}
